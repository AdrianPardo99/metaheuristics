\documentclass[10pt]{article}
\usepackage[utf8]{inputenc}
\usepackage[spanish]{babel}
\usepackage{amsmath}
\usepackage{amsfonts}
\usepackage{amssymb}
\usepackage{graphics}
\usepackage{graphicx}
\usepackage[left=2cm,right=2cm,top=2cm,bottom=2cm]{geometry}
\usepackage{imakeidx}
\makeindex[columns=3, title=Alphabetical Index, intoc]
\usepackage{listings}
\usepackage{xcolor}
\usepackage{multicol}
\usepackage{changepage}
\usepackage{float}
\usepackage{cite}
\usepackage{url}
\usepackage{hyperref}
\usepackage{pdflscape}
\usepackage[document]{ragged2e}
\hypersetup{
    colorlinks=true,
    linkcolor=blue,
    filecolor=magenta,
    urlcolor=blue,
}

\definecolor{codegreen}{rgb}{0,0.6,0}
\definecolor{codegray}{rgb}{0.5,0.5,0.5}
\definecolor{codepurple}{rgb}{0.58,0,0.82}
\definecolor{backcolour}{rgb}{0.95,0.95,0.92}

\lstdefinestyle{mystyle}{
    backgroundcolor=\color{backcolour},
    commentstyle=\color{codegreen},
    keywordstyle=\color{magenta},
    numberstyle=\tiny\color{codegray},
    stringstyle=\color{codepurple},
    basicstyle=\ttfamily\footnotesize,
    breakatwhitespace=false,
    breaklines=true,
    captionpos=b,
    keepspaces=true,
    numbers=left,
    numbersep=5pt,
    showspaces=false,
    showstringspaces=false,
    showtabs=false,
    tabsize=3
}
\def\fillandplacepagenumber{%
 \par\pagestyle{empty}%
 \vbox to 0pt{\vss}\vfill
 \vbox to 0pt{\baselineskip0pt
   \hbox to\linewidth{\hss}%
   \baselineskip\footskip
   \hbox to\linewidth{%
     \hfil\thepage\hfil}\vss}}
\lstset{style=mystyle}

\title{Centro de Investigación en Cómputo\\Instituto Politécnico Nacional\\Metaheurísticas\\Actividad No. 13\\ Solución de problemas mediante Algoritmos Genéticos GA\\Curso impartido por: Dra Yenny Villuendas Rey}

\author{Adrian González Pardo}

\date{\today}

\newcommand\tab[1][1cm]{\hspace*{#1}}

\begin{document}
\maketitle
\section{Ventajas y Desventajas de GA}
\begin{center}
  \begin{tabular}{|p{6cm}|p{6cm}|}
    \hline
    Ventajas & Desventajas \\
    \hline
    Permite realizar multiples busqueda de soluciones a los problemas & Puede que el metodo de mutación o cruza seleccionado puede que no ayude a encontrar una buena solución\\
    \hline
    Esta bioinspirado en la genética y en la selección natural (Darwinismo) & Puede que el que en la aplicación en alguna etapa del GA ya no avance \\
    \hline
    Es una heurística poblacional & Puede que genere bastante uso de recursos en memoria y procesamiento \\
    \hline
    Es posible el trabajar soluciones de formas paralelizables o distribuidas&Puede ser dificil de implementar\\
    \hline
  \end{tabular}
\end{center}
\section{Genotipo vs Fenotipo}
\textbf{Genotipo:} es una representación en cadenas de bits en la cual generalmente es trabajada para generar un nuevo individuo en el algoritmo.
\\
\textbf{Fenotipo:} es la representación que tiene la cadena de bits en el ambito del problema, es decir, la cadena de bits puede representar números reales $\mathbb{R}$, números enteros $\mathbb{Z}$, valores binarios $\{0,1\}$, indices de la solución a algún problema.
\section{Modelo Generacional vs Estacionario}
\subsection{Semejanzas}
\begin{itemize}
  \item Ambos generan en cada iteración nuevas respuesta a analizar
  \item Ambos remplazan a la generación anterior (Con precaución de como son seleccionados y de como trabajan en la siguiente iteración).
  \item Ambos realizan conceptualmente las mismas operaciones (Solo que de diferente manera a la hora de selección y cruza).
\end{itemize}
\subsection{Diferencias}
\begin{itemize}
  \item El modelo generacional crea una nueva población completa, mientra que el modelo estacionario escoje dos partes de la población de acuerdo al muestreo que realice y sobre ellos aplica los operadores genéticos.
  \item El modelo generacional remplaza completamente a la anterior generación, mientras que el modelo estacionario remplaza a nos $N$ cromosas con los $N$ descendientes de la población inicial.
  \item El modelo generacional tiene un remplazo aleatorio, mientras que el modelo estacionario remplaza a los $N$ peores.
  \item El modelo generacional teorícamente realiza una excesiva exploración lo cual no garantiza que tenga una convergencia en un optimo local (Explora espacialmente las regiones de solución), el modelo estacionario realiza una excesiva explotación lo cual converge en un optimo local (Busca mejorar al mejor individuo).
\end{itemize}
\section{Operadores de selección}
\subsection{Muestreo Aleatorio Universal (SUS)}
El modelo de selección se trata de seleccionar varios elementos de acuerdo al tamaño de la población $N$ dado que se obtiene una circunferencia donde existen varias flechas de selección dadas por \(\displaystyle \frac{2\pi}{N}\) donde se busca girar todas las flechas uniformemente de tal manera que el giro y la selección se determina por \(\displaystyle \left[rand\;\times\;\left(\frac{2\pi}{N}\right)\right]\) donde \(\displaystyle rand\sim U(0,1)\), el cual al ser una parte universal elimina el sesgo probabilistico a la hora de seleccionar los elementos.

\subsection{Torneo}
Este modelo se trata de ir realizando un $k$ número de arreglos permutados de forma aleatoria de tamaño $N$ que es el tamaño dimensional de la respuesta de tal modo en que se busca seleccionar el valor indice de aquellos datos que esten mejor evaluados en el torneo, gráficamente el torneo se ve de la siguiente manera:\\
\(\displaystyle k=2\), entonces \(\displaystyle [0.5,1,4,0.1]\), la generación aleatoria, generaria lo siguiente por ejempo:\\
\begin{center}
  \begin{tabular}{|c|c|c|c|}
    \hline
    2&1&4&3\\
    \hline
    4&2&3&1\\
    \hline
  \end{tabular}\\
  \textit{Se accedera a los valores de cada subindice del arreglo en su valor y se evaluara el menor valor vía indices es decir de arriba hacia abajo para seleccionar al mejor.}\\
  \begin{tabular}{|c|c|c|c|}
    \hline
    4&1&4&1\\
    \hline
  \end{tabular}
\end{center}

\subsection{Proporcional}
Para este tipo de selección se usa la función
\[p_{i}=\cfrac{f_{i}}{\sum_{j=1}^{N}f_{j}}\]
Donde:\\
\(\displaystyle p_i\) es la probabilidad de ser seleccionado \(\displaystyle p_{i}\)\\\vspace{0.25cm}
\(\displaystyle f_{i}\) es la función de fitness para ese valor de \(\displaystyle i\)\\\vspace{0.25cm}
Por tanto al obtener esto tendremos un pequeño sesgo probabilistico a la hora de seleccionar un área por tanto definiremos al error selectivo como:
\[n_{i}=p_i\;\times\;N\;=\;\left\lfloor\;N\;\times\;\cfrac{f_{i}}{\sum_{j=1}^{N}f_{j}}\right\rfloor\]
En esta ecuación nos podremos percatar de que redondeamos el error a un valor entero, lo que significa que cada individuo podría perderse cualquiera que sea su valor relativo de aptitud.
\subsection{Por Ruleta}
Para determinar este modelo se considera que la probabilidad para generar la circunferencia con ayuda de la ecuación \(\displaystyle p_{i}\) de la proporcional pero con la ayuda de la selección de 1 elemento a diferencia de como lo hace el Muestreo Universal que selecciona K elementos.

\subsection{Por Emparejamiento Variado Inverso (NAM)}
Un padre se escoge aleatoriamente, para el otro selecciona Nnam padres y escoge el más lejano al primer (\(\displaystyle Nnam=3,5,\cdots\)). Está orientado a generar diversidad.
\section{Estructuras de datos necesarias para la implementación}
Para obtener una mejor selección de nuestras soluciones es necesario hacer notar que muchas de estas soluciones son acompañadas del valor indice podemos pensar en un comportamiento de una cola circular de modo en que podamos realizar permutaciones y giros de acuerdo a como nos plasca, por otro lado tambien podemos pensar en hacer uso de arreglos de dimensión $k$ donde $k\leq N$ de modo en que podemos crear una cantidad de arreglos y permutarlos de forma aleatoria para implementar grantidad de nuevos elementos, al menos para lograr implementar una selección por torneo, mientras que por otro lado podemos el obtener la cola donde cada indice sea un valor a la solución total podemos pensar en que este nos ayuda a implementar las funciones de probabilidad de modo que podemos ahorrarnos bastante tiempo a la hora de realizar las evaluaciones.
\section{Implementación}
\lstinputlisting[language=Ruby]{seleccion.rb}


\end{document}
